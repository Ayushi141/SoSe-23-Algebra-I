\documentclass{tufte-handout}

\usepackage{amsmath}
\usepackage{graphicx}
\setkeys{Gin}{width=\linewidth,totalheight=\textheight,keepaspectratio}

\usepackage{booktabs}
\usepackage{units}
\usepackage{fancyvrb}
\fvset{fontsize=\normalsize}
\usepackage{multicol}
\usepackage{lipsum}
\PassOptionsToPackage{dvipsnames}{xcolor}
\usepackage{xcolor}
\usepackage{amsmath, amsthm, thmtools}
\usepackage{amssymb}
\usepackage{cleveref}
\usepackage{csquotes}
\geometry{
  marginparwidth=60mm % width of margin notes
}
%%%%%%%%% mathematical bold  %%%%%%%%%%%%%%%%%

\newcommand{\bA}{\mathbb{A}}
\newcommand{\bB}{\mathbb{B}}
\newcommand{\bC}{\mathbb{C}}
\newcommand{\Cs}{\bC^\times}
\newcommand{\bD}{\mathbb{D}}
\newcommand{\bE}{\mathbb{E}}
\newcommand{\F}{\mathbb{F}}
\newcommand{\bF}{\mathbb{F}}
\newcommand{\bG}{\mathbb{G}}
\newcommand{\bH}{\mathbb{H}}
\newcommand{\bI}{\mathbb{I}}
\newcommand{\bJ}{\mathbb{J}}
\newcommand{\bK}{\mathbb{K}}
\newcommand{\bL}{\mathbb{L}}
\newcommand{\bM}{\mathbb{M}}
\newcommand{\N}{\mathbb{N}}
\newcommand{\bO}{\mathbb{O}}
\newcommand{\bP}{\mathbb{P}}
\newcommand{\bp}{\mathbb{p}}
\newcommand{\Q}{\mathbb{Q}}
\newcommand{\R}{\mathbb{R}}
\newcommand{\bS}{\mathbb{S}}
\newcommand{\bT}{\mathbb{T}}
\newcommand{\bU}{\mathbb{U}}
\newcommand{\bV}{\mathbb{V}}
\newcommand{\bW}{\mathbb{W}}
\newcommand{\bX}{\mathbb{X}}
\newcommand{\bY}{\mathbb{Y}}
\newcommand{\Z}{\mathbb{Z}}

%%%%%%%%% calligraphic %%%%%%%%%%%%%%%%%%%%%%%

\newcommand{\mc}[1]{\mathcal{#1}}
\newcommand{\cA}{\mathcal{A}}
\newcommand{\cB}{\mathcal{B}}
\newcommand{\cC}{\mathcal{C}}
\newcommand{\cD}{\mathcal{D}}
\newcommand{\cE}{\mathcal{E}}
\newcommand{\cF}{\mathcal{F}}
\newcommand{\cG}{\mathcal{G}}
\newcommand{\cH}{\mathcal{H}}
\newcommand{\cI}{\mathcal{I}}
\newcommand{\cJ}{\mathcal{J}}
\newcommand{\cK}{\mathcal{K}}
\newcommand{\cL}{\mathcal{L}}
\newcommand{\cM}{\mathcal{M}}
\newcommand{\cm}{\mathcal{m}}
\newcommand{\cN}{\mathcal{N}}
\newcommand{\cO}{\mathcal{O}}
\newcommand{\cP}{\mathcal{P}}
\newcommand{\cQ}{\mathcal{Q}}
\newcommand{\cR}{\mathcal{R}}
\newcommand{\cS}{\mathcal{S}}
\newcommand{\cT}{\mathcal{T}}
\newcommand{\cU}{\mathcal{U}}
\newcommand{\cV}{\mathcal{V}}
\newcommand{\cW}{\mathcal{W}}
\newcommand{\cX}{\mathcal{X}}
\newcommand{\cY}{\mathcal{Y}}
\newcommand{\cZ}{\mathcal{Z}}

%%%%%%%%% mathematical fraktur  %%%%%%%%%%%%%%

\newcommand{\mf}[1]{\mathfrak{#1}}
\newcommand{\fa}{\mathfrak{a}}
\newcommand{\fb}{\mathfrak{b}}
\newcommand{\fc}{\mathfrak{c}}
\newcommand{\fA}{\mathfrak{A}}
\newcommand{\fB}{\mathfrak{B}}
\newcommand{\fC}{\mathfrak{C}}
\newcommand{\fD}{\mathfrak{D}}
\newcommand{\fE}{\mathfrak{E}}
\newcommand{\fF}{\mathfrak{F}}
\newcommand{\fG}{\mathfrak{G}}
\newcommand{\fH}{\mathfrak{H}}
\newcommand{\fI}{\mathfrak{I}}
\newcommand{\fJ}{\mathfrak{J}}
\newcommand{\fK}{\mathfrak{K}}
\newcommand{\fL}{\mathfrak{L}}
\newcommand{\fm}{\mathfrak{m}}
\newcommand{\fN}{\mathfrak{N}}
\newcommand{\fO}{\mathfrak{O}}
\newcommand{\fp}{\mathfrak{p}}
\newcommand{\fQ}{\mathfrak{Q}}
\newcommand{\fq}{\mathfrak{q}}
\newcommand{\fR}{\mathfrak{R}}
\newcommand{\fS}{\mathfrak{S}}
\newcommand{\fT}{\mathfrak{T}}
\newcommand{\fU}{\mathfrak{U}}
\newcommand{\fV}{\mathfrak{V}}
\newcommand{\fW}{\mathfrak{W}}
\newcommand{\fX}{\mathfrak{X}}
\newcommand{\fY}{\mathfrak{Y}}
\newcommand{\fZ}{\mathfrak{Z}}

%%%%%%%%%    math operators    %%%%%%%%%%%%%%%

\newcommand{\RP}{\mathbb{RP}^2}
\newcommand{\gen}[1]{\langle #1\rangle}
\newcommand{\Id}{\mathrm{Id}}
\DeclareMathOperator{\im}{im}
\DeclareMathOperator{\ggt}{ggT}
\DeclareMathOperator{\kgv}{kgV}
\DeclareMathOperator{\Aut}{Aut}
\DeclareMathOperator{\Hom}{Hom}
\DeclareMathOperator{\Iso}{Iso}
\DeclareMathOperator{\ord}{ord}
\DeclareMathOperator{\rad}{rad}
\DeclareMathOperator{\GL}{GL}
\DeclareMathOperator{\SL}{SL}
\DeclareMathOperator{\UT}{UT_3(\R)}
\DeclareMathOperator{\id}{id}
\DeclareMathOperator{\sgn}{sgn}
\DeclareMathOperator{\modn}{mod}
\DeclareMathOperator{\stab}{Stab_G}
\DeclareMathOperator{\Th}{Th}
\DeclareMathOperator{\On}{On}
\DeclareMathOperator{\Spec}{Spec}
\DeclareMathOperator{\coker}{coker}

%%%%%%%%%    further commands  %%%%%%%%%%%%%%%

\DeclareMathOperator{\nil}{nil}
\DeclareMathOperator{\Jac}{Jac}
\setlength{\headheight}{13.59999pt}

%%%%%%%%%    thmtools environments  %%%%%%%%%%%%%%%

\declaretheoremstyle[
shaded={
    rulecolor=Lavender!35,
    rulewidth=2pt,
    bgcolor=Lavender!25},
spaceabove=6pt, spacebelow=6pt,
headfont=\normalfont\bfseries, headindent=\parindent,
notefont=\mdseries, notebraces={(}{)},
bodyfont=\normalfont,
postheadspace=0.5em]{lav}

\declaretheoremstyle[
shaded={
    rulecolor=RoyalBlue!12,
    rulewidth=2pt,
    bgcolor=RoyalBlue!8},
spaceabove=6pt, spacebelow=6pt,
headfont=\normalfont\bfseries, headindent=\parindent,
notefont=\mdseries, notebraces={(}{)},
bodyfont=\normalfont,
postheadspace=0.5em]{blue}

\declaretheoremstyle[
shaded={
    rulecolor=RoyalBlue!8,
    rulewidth=2pt,
    bgcolor=RoyalBlue!5},
spaceabove=6pt, spacebelow=6pt,
headfont=\normalfont\bfseries, headindent=\parindent,
notefont=\mdseries, notebraces={(}{)},
bodyfont=\normalfont,
postheadspace=0.5em]{lightblue}

\declaretheorem[
name=Definition, 
style=lav, 
numberwithin=section,
refname={definition, definitions},
Refname={Definition, Definitions}
]{definition}

\declaretheorem[
name=Theorem, 
style=blue, 
sibling=definition
]{theorem}

\declaretheorem[
name=Lemma, 
style=lightblue, 
sibling=definition
]{lemma}

\declaretheorem[
name=Proposition, 
style=lightblue, 
sibling=definition
]{proposition}

\declaretheorem[
name=Corollary, 
style=lightblue, 
sibling=definition
]{corollary}

\declaretheorem[
name=Exercise, 
style=lightblue, 
sibling=definition
]{exercise}

\declaretheorem[
name=Note, 
style=lav, 
sibling=definition
]{note}

\title{Lecture Notes Algebra I - Commutative Algebra\thanks{Held by Dr. Andreas Mihatsch at University of Bonn in summer 2023.}}
\author[Ayushi Tsydendorzhiev]{Ayushi Tsydendorzhiev}

\setlength{\headheight}{14.0pt}
\usepackage{enumitem}
\usepackage{stmaryrd}

\newcommand{\QT}{\mathbb{Q}[T]}
\newcommand{\ZT}{\mathbb{Z}[T]}
\newcommand{\bigslant}[2]{{\raisebox{.1em}{$#1$}\left/\raisebox{-.1em}{$#2$}\right.}}
\setcounter{MaxMatrixCols}{11}
\DeclareRobustCommand\longtwoheadrightarrow
     {\relbar\joinrel\twoheadrightarrow}
     
\begin{document}

\maketitle

\begin{abstract}
\noindent These are my notes for the Algebra I class. Regretfully, I'm not very good at math and these notes will be at times lengthy and/or wrong. But, well, over time I found that the best way for me personally to learn was to write everything down and explain it to myself. Maybe one day someone else will find it useful.
\end{abstract}

\tableofcontents

\newpage

\section{Intro to Commutative Algebra}\label{sec:page-layout}
\subsection{Rings and Ideals}\label{sec:headings}

In these lecture notes, a ring is always commutative and unitary (has element 1).
\begin{definition}
    \textbf{Ideal} $\fa =$ abelian subgroup, such that $\forall r\in R, a\in \fa : ra \in \fa$.  
\end{definition}

\begin{definition}\label{IdealGenerated}
    Let $S\subseteq A$ be a subset of $A$. Then the \textbf{ideal, generated by $S$} is defined as 
    $$(S):= \bigcap_{\substack{S\subseteq \fa \subseteq A \\ \text{$\fa$ is an ideal}}} \fa$$
\end{definition}
\marginnote[-30pt]{Could it be a closure operator on sets?}

\begin{lemma}[Equivalent to \cref{IdealGenerated}]
    Let $A$ be a ring, and let $S\subseteq A$ be a subset. Then we have
    $$(S)=\sum_{s\in S} As = \{ \sum a_s s \mid a_s \in A \text{ and finitely many } a_s \neq 0 \}$$
\end{lemma}
\textit{Proof:} Let $\fb$ be the right-hand side. It is an additive subgroup, since 
$$ (\sum_{s\in S} a_s s)^{-1} = \sum_{s\in S} a_s^{-1} s \in \fb$$
and 
$$ \sum_{s\in S} a_s s + \sum_{s\in S} b_s s = \sum_{s\in S} (a_s+b_s) s \in \fb.$$
It is also closed under multiplication, thus $\fb$ is an ideal. Since $1s\in \fb$ it follows $S\subseteq \fb$ and hence by definition $(S) \subseteq \fb$.

Conversely, let $\fa \subseteq A$ be an ideal such that $S\subseteq \fa$. Then from the ideal properties we get $as \in \fa$ for all $s \in S$ and thus $\sum_{s\in S} a_s s \in \fa$ for lal finite sums. Therefore $\fb \subseteq \fa$ and finally $\fb \subseteq S$.

\begin{definition}
    Given any ring $A$, we can construct \textbf{polynomial rings} $A[T]$ as formal sums over $A$ in a single variable $T$:
    $$A[T]:=\oplus_{i=0}^\infty A T^i = \{ \sum_{i=0}^n a_i T^i \mid n\geq 0, a_i \in A, a_n \neq 0\}.$$
\end{definition}

\begin{definition}
    Given a ring $A$ and an ideal $\fa \subseteq A$, the additive abelian quotient group $A/\fa$ endowed with the multiplication 
    $$(a+\fa)(b+\fa):=ab+\fa$$
    forms a ring which we call a \textbf{quotient ring} of $A$.
\end{definition}

\marginnote[20pt]{Math consists of learning vocabularies.}
\begin{definition}
    Let A be a ring, then
    \begin{enumerate}
        \item $x\in A$ is \textbf{nilpotent}, if $x^n = 0$ for some $n\in \N$. $A$ is \textbf{reduced} if $0$ is the only nilpotent element.
        \item $x\in A$ is a \textbf{zero divisor}, if there exists $y\in A$ such that $xy= 0$. $A$ is an integral domain if $0$ is the only zero divisor and $A\neq 0$.
        \item $x\in A$ is a \textbf{unit}, if there exists $y\in A$ such that $xy=1$. The set of all units in $A$ is denoted by $A^\times$ and forms a multiplicative group.
    \end{enumerate}
\end{definition}

\marginnote[-40pt]{Integral domains are always reduced. On the other hand, $Z[X,Y]/(XY)$ is reduced but not integral.}

\begin{lemma}
    Consider a map $\phi: A \rightarrow A, a \mapsto xa$ for a fixed $x\in A$.
    It follows that
    $\phi$ bijective $\iff$ $\phi$ surjective $\iff x\in A^\times$.
\end{lemma}
\textit{Proof:} $\phi$ bijective implies $\phi$ surjective. $\phi$ surjective implies $\exists a\in A : xa = 1 \implies x$ is a unit $\implies x\in A^\times$. Conversely, $x\in A^\times \implies \exists a \in A : xa = 1 \implies \forall b \in A : xab = 1b = b \implies xa = 1 = xa' \iff a=a' \implies \ker \varphi$ is trivial. 

\begin{lemma}
    If $A$ is reduced then $A[T_i, i\in I]$ is reduced as well for any index set $I$.
\end{lemma}

\begin{definition}
    Let $A$ be a ring. Define \textbf{nilradical} of $A$
    $$\nil(A)=\{a\in A \mid a \text{ nilpotent}\}.$$
\end{definition}

\begin{proposition}[Properties of nilradical]~

    \begin{enumerate}
        \item $\nil(A)$ is an ideal,
        \item $A/\nil(A)$ is reduced,
        \item \textbf{Universal property of nilradicals:} For any reduced ring $B$, any ring map $\phi: A\rightarrow B$ factors through $A/\nil(A)$.
    \end{enumerate}     
\end{proposition}

\marginnote[-80pt]{Kernels are ideals; nilradicals are ideals too. If the codomain ring is reduced then $\nil(A) \subseteq \ker(\varphi)$. So dimension has to do with certain properties of \enquote{flatness}.}

\textit{Proof:} 
\begin{enumerate}
    \item Let $a, b\in \nil(A) \implies a^n = 0$. Then $\forall x \in A: (xa)^n = x^n a^n=0$. Furthermore, $(a+b)^{n+m-1} = \sum_{i=0}^{n+m-1} \binom{n+m-1}{i}x^{n+m-1}y^i = 0$, since either $(n+m-1-i)\geq n$ or $i\geq m$.
    \item Let $\overline{x} = x + \nil(A)$. Then $\overline{x}$ is nilpotent iff $x\in \nil(A) \implies \overline{x} = 0$.
    \item Let $B$ be reduced and let $\varphi: A \rightarrow B$ be a ring map. If $x^n=0$ for $x\in\nil(A)$, then $\varphi(x)^n=0$, so $\varphi(x)=0$ since $B$ is reduced. In other words, $\nil(A)\subseteq \ker(\varphi)$, hence $\ker(\varphi)$ factors through $A / \nil(A)$ according to the universal property of the quotients.
\end{enumerate}

\subsection{Fields}

This chapter was very light on content. 

\begin{definition}
    A ring $A$ is a field if $A\neq 0$ and $A^\times = A \setminus \{0\}$.
\end{definition}
\marginnote[-20pt]{Non-zero and all elements are invertible.}

\begin{lemma}
    $A$ is a field $\iff$ the only ideals are $\{0\}$ and $A$.
\end{lemma}

\begin{definition}
     An ideal $\fm$ is \textbf{maximal} if $\fm \neq A$ and there is no ideal $\fa$ such that $\{0\} \subset \fa \subset \fm$.
\end{definition}

\begin{corollary}
    Let $A$ be a ring. An ideal $\fm$ is maximal $\iff A/\fm$ is a field.
\end{corollary}

\subsection{Principal ideal domains}

\begin{definition}
    An integral domain $A$ is a \textbf{principal ideal domain} (PID) if every ideal $\fa \subset A$ is \textbf{principal}, i. e. of the form $\fa = (f)$ for some $f\in A$.
\end{definition}
\marginnote[-40pt]{$\bC[\varepsilon]/(\varepsilon^2)$ is not an integral domain, but every ideal is principal (there are only three).} 

\begin{definition}
    A ring is a \textbf{principal ideal ring} if every ideal is principal.
\end{definition}

\begin{definition}
    Let $A$ be an integral domain. Then $p\in A$ is \textbf{prime} if $p$ is not the zero element or not a unit and $p\mid ab$ implies $p\mid a$ or $p\mid b$.
\end{definition}

\begin{theorem}
    In PIDs, prime factorization theorem holds. 
\end{theorem}
\marginnote[-20pt]{In unique factorization domains factorization in \textbf{irreducible} elements holds. The condition of being a prime element is stronger then being irreducible. For example, $3$ is irreducible but not prime in $\Z[\sqrt{-5}]$.}

\subsection{Power series}

\begin{definition}
    Let $A$ be a ring. Then 
    $$A\llbracket T \rrbracket := \{ \text{infinite series } \sum_{i=0}^\infty a_iT^i \mid a_i \in A \} \cong A^{\Z_{\geq 0\}}}.$$
\end{definition}

\begin{proposition}
    Let $A$ be a ring. Then $f\in A\llbracket T \rrbracket^\times$ if and only if $a_0\in A^\times$.
\end{proposition}
\newpage

\begin{exercise}
    Show that a prime $p\neq 3$ is of the form $p=x^2-xy+y^2$ iff $p\equiv 1 \pmod{3}$.
\end{exercise}

\textit{Proof:} %First the only if part. The quadratic residue classes mod $3$ are:
%$$0^2=0 \qquad 1^2=1 \qquad 2^2=1$$
%Therefore $x^2+y^2=0,1,2 \mod 3$. If $x^2+y^2=2\mod 3$, then $xy=4 \mod 3 = 1\mod 3$ and the whole term is equal to $1\mod 3$. And if $x^2+y^2 = 0 \mod 3$ then the $p$ is divisble by $3$. So the only option really is $1 \mod 3$.
Observe that $p\equiv x^2-xy+y^2 \equiv x^2+2xy+y^2 \mod 3$. This implies $p\equiv(x+y)^2 \mod 3$. The quadratic residue classes $\mod 3$ are $0^2=0, 1^2=1, 2^2=1$, which implies either $p=3$ or $p\equiv 1 \mod 3$.

\textit{Goal:} Now the hard part. We want to look at fibers of map
\begin{align*}
\Spec(\varphi): \Spec(\Z[\zeta])&\longrightarrow \Spec(\Z) \\
\fa &\longmapsto \varphi^{-1}(\fa)    
\end{align*}
We know that for any \marginnote[-45pt]{I'm not sure what exactly this map is. I think it's the inclusion map, e. g. it maps $\fm$ to $(p)$ such that $(p) \subseteq \fm$. In a sense $\Z[\zeta]$ has a certain \enquote{torsion} which allows for bigger, stronger maximal ideals than in $\Z$.}
$\fm \in \Spec(\Z[\zeta])$, the intersection $\fm \cap \Z = (p)$. Specifically for $\Z[\zeta]$, we know $\Z[\zeta_3] \cong \Z[T]/(T^2+T+1)$ because minimal polynomial of $\zeta$ is $m_\zeta = T^2+T+1$. 

\textit{Prime Ideals:} Given $\Z[T]/(T^2+T+1)$, what prime ideals can exist there? The answer is partially known. It's either $(p)$ for $p$ prime, or $(p, h_i)$ for $h_i$ lift of an irreducible factor of $T^2+T+1$. So we should think hard about the question of irreducibility of $m_\zeta$.

\textit{Irreducibility of $m_\zeta$:} If $p=3$, then 
\begin{align*}
    \{\fm \subset \Z[T]/(m_\zeta) \mid \fm \cap \Z = (3)\} & = \{\fm \subset A \mid (3) \subseteq \fm\} \\
    & = \Spec(\bigslant{\Z[T]/(m_\zeta))}{(3)})\\
    & = \Spec(\Z[T]/(m_\zeta, 3)\\
    & = \Spec(\bigslant{\bF_3[T]}{(m_\zeta \mod (3))}) \\
    & = \{(h_i)\mid \text{ irreducible factors $h_i\in \bF_3[T]$ of $m_\zeta$}\} \\
    & = \{(p,\widetilde{h_i}) \mid \widetilde{h_i} = \text{ lift of $h_i$ to $\Z[T]$}\}.
\end{align*}
\marginnote[-130pt]{What kind of lift? Basically we remember something along the lines of the 3rd isomorphism theorem, stating $$\frac{A/\fm}{\fm/(p)} = \frac{A}{(p)},$$ but in this case it's more of $$\frac{A/\fm}{(p)}=\frac{A/(p)}{\fm/(p)}.$$}
This schema works for any $p$, so essentially we are interested in factorizations of $m_\zeta$ over any $\bF_p[T]$. \marginnote[-20pt]{If $m_\zeta$ is irreducible, then the fiber is given by $(p)$ only, since $\bF_p[T]/(m_\zeta)$ is a field.} 
By a straight-forward calculation, have $\fm_{\zeta}=(T+2)^2$.\\
If $p\equiv 1 \mod (3)$ then $\bF_p^\times$ has order $p-1$ and as such has a non-trivial third root of unity if and only if $3 \mid (p-1)$. This obviously holds, which means $\fm_\zeta = (T-\alpha)(T-\alpha^2)$.\\
This property doesn't hold if $p\equiv 2 \mod 3$, implying $\fm_\zeta$ irreducible, otherwise it wouldn't be minimal.\\
Summarizing the above, have
$$\Spec(\Z[\zeta]) = \coprod_{0 \text{ or } p \text{ prime} } \begin{cases}
    (0) & \\
    (3, \zeta +2) & $p=3$ \\
    (p, \zeta-\alpha),(p,\zeta-\alpha^2) & p \equiv 1 \mod (3)\\
    (p) & p\equiv 2 \mod (3) 
\end{cases}$$
\marginnote[-70pt]{Is $0$ even prime?}
Now observe that $\Z[\zeta]$ is a PID. Let $(\pi)\in \Spec(Z[\zeta])$ with $\pi$ prime. Now skipping some computations we claim $\pi = p$ by norm function and unique decomposition theorem, which implies $\pi = x+iy \in \Z[\zeta]$ such that $N(\pi)=x^2-xy+y^2=p$.

Basically the whole trick is: observe that norm $N(x)$ defined on $Z[\zeta]$ has some nice formula such as $x^2+ny$. Since maximal ideals in $\Z$ correspond to prime numbers, we can try to extend $\Z$ such that these prime ideals are generated by some \enquote{smaller} elements, such that its norm equals precisely to $p$. By the virtue of our coordinates being integer we prove the claim. 

\begin{exercise}
    Let $A$ be a principal ideal domain that is not a field, let $\fm \subset A$ be a maximal idea. Prove that $\fm^n / \fm^{n+1}$ is a one-dimensional vector space over $A/\fm$ for any $n\geq 0$.
\end{exercise}
\textit{Proof:} That's a lot to unpack. Start with definition for $\fm^n / \fm^{n+1}$.

$$\fm^n / \fm^{n+1}= (a)^n / (a)^{n+1}$$

where $a$ is generator of $\fm$. As such, any element in $(a)^n$ is of the form $ca^n, c \in A$. Now if we look at the quotient as if it were a graded ring, \enquote{going up} one degree to $a^{n+1}$ annihilates element to $0$, which happens precisely if you multiply by some element $x\in (a) \implies (a)\cdot (a)^n = 0 \in \bigslant{(a)^n}{(a)^{n+1}}$. So it's natural to describe $(a)^n$ as a one-dimensional vector space over $A/(a)$.
\marginnote[-30pt]{This is what \textit{associated graded ring} does. Essentially it's a direct sum $\bigoplus_{n=0}^\infty (a)^n/(a)^{n+1}$.}
If $A$ is a field then $\fm=(0)$ and as such it is $0$-dimensional over $A$.

\begin{exercise}
    Compute all fibres of $\Spec (\ZT) \rightarrow \Spec(\Z)$.
\end{exercise}

\textit{Proof:} \marginnote[-5pt]{This part we've already seen.}Assume that $\fp\cap \Z = (p)$ for some prime $p\in \Z$. Then $\overline{\fp}:= \fp / p\ZT$ is a prime ideal in $\F_p[T]$. Since $\F_p[T]$ is a PID, $\overline{\fp}=(\overline{f})\in \F_p[T]$. Hence we have
\begin{itemize}
    \item $\fp = (p)$ if $p=0$,
    \item $\fp = (p,f)$ if $\overline{\fp}=(\overline{f})$, where $f$ is any lift of $\overline{f}= f \mod p$. 
\end{itemize}

Assume $\fp \cap \Z = (0)$. \marginnote[-5pt]{This part we haven't seen. It uses localization.} Consider $\fq=\fp \QT$, i. e. the ideal in $\QT$ generated by elements in $\fp$. We claim that $\fq$ is a prime ideal in $\QT$. 

If $\fq$ isn't prime and $1\in\fq$, \marginnote[-5pt]{The $\fq\neq \QT$ part.} then we can write $1=\sum f_i a_i$ with $f_i\in \QT, a_i\in\fp$. Let $0\neq m \in \Z$ be the common denominator of all coefficients of all $f_i\in \QT$. Then $mf_i\in \Z[T]$ for all $i=1,\ldots,n$, hence $m1=\sum (mf_i)a_i\in\fp$ which yields the contradiction with $\fp \cap \Z = (0)$. This means $1\notin \fq$ and thus $\fq \neq \QT$.
% mistake with f, g, h in script

Let $gh\in\fq$ for some $g,h\in \QT$.\marginnote[-5pt]{The $\fq$ is prime part.} Then we can write $gh=\sum f_ia_i$ with $f_i \in \QT$ and $a_i \in\fp$. Now choose common denominator $0\neq m\in \Z$ such that $mg,mh, mf_i\in\ZT$. Then we lift $g$ and $h$ to $\Z$ and observe 
$$mg\cdot mh = m \sum \underbrace{(mf_i)}_{\in \ZT}\underbrace{a_i}_{\in\fp\subseteq \ZT} \in \fp$$ 
and by the prime ideal property either $mf\in \fp$ or $mg \in \fp$. Multiplying by $m^{-1}\in\Q$, we get either $g\in\fq$ or $h\in\fq$, implying that $\fq$ is prime.

Since $\QT$ is a PID, we can write $\fq = (h)$ for some irreducible $h\in \QT$. We can lift it to some $mh\in \ZT$. Further factoring out the gcd of all coefficients, we can assume that $mh$ is primitive. From \textit{Gauss's lemma} it follows: if $mh\in\ZT$ is primitive and $f \in\ZT$, then $mh\mid f$ in $\ZT$ iff $mh\mid f$ in $\QT$. \marginnote[-10pt]{In other words: if a primitive polynomial $mh$ divides polynomial $f$ in $\ZT$, it does so in $\QT$.} 

As a consequence, we have $\fq \cap \ZT = (h) \in \ZT$ with irreducible and primitive $h$. Later we show $\fq \cap \ZT = \fp$.

\begin{note}
     What do we actually do here? At first, we look at the intersection between $\ZT$ and the smaller ring $\Z$. We find out it's empty (zero). What do we do know? We investigate the bigger fraction field of $\ZT$, its localization at $0$ and look at what kind of ideal does $\QT \fp$ generate. In some sense since our first, superficial method didn't work we localize around $0$ and dig deeper at what does $\fp$ actually generate there. From there on we find out that $\QT\fp$ generates another prime ideal $\fq$, which is generated by a single element $h\in \QT$. By Gauss's lemma (sheer luck) this element also is in $\ZT$, implying $\fp = (h)$. Insane, right? At first, we know nothing about prime ideals. But we know about their images in $\Z$. And this information is enough to hunt them down in two different realms.
\end{note}

\begin{exercise}
    Assume $A$ is Noetherian. Prove $A[[T]]$ is Noetherian (Hilbert's Basis Theorem).
\end{exercise}
\marginnote[-30pt]{Noetherian property is stable by passage to finite type extensions and localization.}
\textit{Proof:} As a reminder, Noetherian $\iff$ every ideal is finitely generated. Let $\fa \in A[[T]]$ be a ideal. We show $\fa$ is finitely generated. For each integer $n$, denote
$$I_n=\{a\in A\mid f=ax^n+\textit{higher order terms}\in \fa\}\in A$$
Then we see that $I_0 \subset I_1 \subset \ldots$ stabilizes, as $A$ is Noetherian. Choose $d_0$ such that $I_{d_0}=I_{d_0+1}=\ldots$ For each $d\leq d_0$ choose elements 
$$f_{d,j}\in I\cap (T^d) \qquad j=1\ldots n_d$$
such that if we write $f_{d,j}=a_{d,j}T^d + \textit{higher order terms}$ then $I_d=(a_{d,1}\ldots a_{d,n_d})$.

\textit{Example:} Let $d_0 = 10$. Then we have 
$$I_0 \subset I_1 \subset \ldots \subset I_{10} = I_{11} = I_{12} = \ldots$$
Now choose 

\[
\begin{matrix}
    j= & 1 & 2 & 3 & 4 & 5 & 6 & 7 & 8 & 9 & 10 \\
    \hline
    f_{0,j} & 1 & 2 & 3 & 4 & 5 & 6 & 7 & 8 & 9 & 10 \\
    f_{1,j} & 1 & 2 & 3 & 4 & 5 & 6 & 7 & 8 & 9 & 10 \\
    f_{2,j} & 1 & 2 & 3 & 4 & 5 & 6 & 7 & 8 & 9 & 10 \\
    f_{3,j} & 1 & 2 & 3 & 4 & 5 & 6 & 7 & 8 & 9 & 10 \\
    f_{4,j} & 1 & 2 & 3 & 4 & 5 & 6 & 7 & 8 & 9 & 10 \\
    f_{5,j} & 1 & 2 & 3 & 4 & 5 & 6 & 7 & 8 & 9 & 10 \\
    f_{6,j} & 1 & 2 & 3 & 4 & 5 & 6 & 7 & 8 & 9 & 10 \\
    f_{7,j} & 1 & 2 & 3 & 4 & 5 & 6 & 7 & 8 & 9 & 10 \\
    f_{8,j} & 1 & 2 & 3 & 4 & 5 & 6 & 7 & 8 & 9 & 10 \\
    f_{9,j} & 1 & 2 & 3 & 4 & 5 & 6 & 7 & 8 & 9 & 10 \\
    f_{10,j}& 1 & 2 & 3 & 4 & 5 & 6 & 7 & 8 & 9 & 10 \\
    \hline
\end{matrix}
\]

\section{Intro to Homological Algebra}

In this chapter, $M, N, P$ are $A$-modules.

\subsection{Preliminaries \& Motivation}

\begin{definition}
    A map $f:M\times N \rightarrow P$ is \textbf{bilinear}, if it is linear in each variable separately. $\forall a \in A, m, m'\in M, n, n'\in N:$
    \begin{itemize}
        \item $f(m,n+n')=f(m,n)+f(m,n')$
        \item $f(m,an)=af(m,n)$
        \item $f(m+m',n)=f(m,n)+f(m',n)$
        \item $f(am,n) = af(m,n)$
    \end{itemize}
\end{definition}

Linear maps are completely determined by their action on bases. How can we determine bilinear maps? For free modules it is enough to know their action on all pairs $(v_i,w_j)$, where $v_i$ and $w_j$ are basis vectors for $M$ and $N$. We would gladly extend this case to the bilinear case.

If we were to take the basis of $V\times W$, for example $\R \times \R$, then knowing the action of $f$ on $(1,0)$ and $(0,1)$ is not enough, since $f(1,0)=f(1,0+0) = f(1,0)+f(1,0) \implies f(1,0) = 0$. Turns out the most general way to map linear maps to bilinear maps is by mapping $(v,w)$ to $v\otimes w$.

\subsection{Tensor Product}

\begin{definition} 
    For any two $M,N$ define a pair 
    $$(A\text{-module } T,  \text{bilinear map }g:M\times N \rightarrow T)$$
    as \textbf{the tensor product} of $M$ and $N$ over $A$, if it has the following property:

    Given any $P$ and any $A$-bilinear mapping $f:M\times N \rightarrow P$, there exists a unique $A$-linear mapping $f':T\rightarrow P$ such that $f= f' \circ g$. It always exists and is unique.
\end{definition}

\marginnote[-40pt]{
    In other words, we have a bijection 
    $\{\text{bilinear maps } M\times N \rightarrow P\} 
    \longleftrightarrow
    \{\text{linear maps } M \otimes N \rightarrow P\}.
    $
}

\begin{exercise}
    Show $(M\otimes N) \otimes P \cong M \otimes (N\otimes P)$
\end{exercise}

\textit{Proof:} 

Step 1: Fix $p\in P$, define $\phi_p: M\times N \longrightarrow M \otimes (N\otimes P), (m,n) \longrightarrow m \otimes (n \otimes p)$. This map is bilinear. It induces a linear map $\overline{\phi_p}: M \otimes N \rightarrow  M \otimes (N\otimes P)$.

Step 2: Consider the induced map $\overline{\phi_p}$. It is linear in $p$, meaning $\overline{\phi_{p+p'}}=\overline{\phi_p} + \overline{\phi_{p'}}$, $\overline{\phi_{ap}} = a \overline{\phi_p}$.

Step 3: Since the above is true for all $p\in P$, consider bilinear maps
$$(M\otimes N)\times P \rightarrow M \otimes (N\otimes P)$$
which sends 
$$\Bigg[\Bigl(\sum_i m_i \otimes n_i\Bigl), p \Bigg] \longrightarrow \overline{\varphi_p}(\sum_i m_i \otimes n_i)p$$
It induces a linear map 
$$(M\otimes N)\otimes P \rightarrow M \otimes (N\otimes P)$$
And we're done?
\begin{theorem}
    Important equivalences for modules over $A$
    \begin{itemize}
        \item $A\otimes_A M \cong M$,
        \item $M \otimes N \cong N \otimes M$,
        \item $(M\otimes N) \otimes P \cong M \otimes (N \otimes P)$,
        \item $(\bigoplus_{i\in I} M_i) \otimes N \cong \bigoplus_{i\in I} (M_i \otimes N)$,
        \item $A/\fa \otimes M \cong M/\fa M$ 
    \end{itemize}
\end{theorem}

\subsection{Modules}

In this section we discuss module properties.

\begin{definition}
    Classification of finiteness properties I.

    0. $M$ is free iff $M$ has basis.

    1. $M$ is finitely generated iff $A^{\oplus m} \longrightarrow M \longrightarrow 0$ is exact.

    2. $M$ is finitely presented iff $A^{\oplus n} \longrightarrow A^{\oplus m} \longrightarrow M \longtwoheadrightarrow 0$ is exact.
\end{definition}

$0.$ implies that $f:A^{\oplus m} \longrightarrow M$ is an isomorphism. $1.$ implies that $f:A^{\oplus m} \longrightarrow M$ is surjective., which implies that $M$ is generated by some finite $(m_1\ldots m_m)$ but the kernel of $f$ has some non-trivial part. The caveat is that the basis may not exist, e.g. $\Z = (2,3)$ but minimal generating set is $\emptyset$.
\marginnote[-30pt]{Important disclaimer: This classification doesn't apply to rings. Apparently, it's because the category of rings is not \textit{abelian}.}

\textit{Some examples:}
\begin{itemize}
    \item Finitely generated free module --- $\Z = {1}_\Z, \R^2 = \{(1,0), (0,1)\}_\R$.
    \item Finitely generated non-free module --- $\Z/n\Z = (1)_\Z$, but $1\cdot n = 0$.
    \item Non-finitely generated free module --- $\bigoplus_{i=1}^\infty \Z$.
    \item Non-finitely generated non-free module --- $\Q$ over $\Z$.
\end{itemize}

\begin{definition}
    Classification of finiteness properties II.
    
    Every module has a presentation $M=N/K$. 
    
    0. $M$ is free iff $N$ is finitely generated and $K=0$.
    
    1. $M$ is finitely generated iff $N$ is finitely generated.
    
    2. $M$ is finitely presented iff $N, K$ are finitely generated.

    The $N/K$ quotient is a hidden way to express $\coker(A^{\oplus n}\rightarrow A^{\oplus m})$, so modules can be also thought of in terms of the $A^{\oplus n}\rightarrow A^{\oplus m}$ map.
\end{definition}
\marginnote[-40pt]{There is no simple way to describe rings as cokernels in exact sequences, see margin note above.}

\subsection{Exactness properties}
\begin{itemize}
    \item Tensoring is right-exact 
    \item Localization of rings is exact
    \item Localization of modules is exact
\end{itemize}

\subsection{Universal properties}
\begin{itemize}
    \item Universal property of quotients
    \item Universal property of direct products
    \item Universal property of direct sums
    \item Universal property of polynomial rings
    \item Universal property of tensor products
\end{itemize}

\subsection{Flat modules}
\begin{definition}
    An $A$-module $M$ is \textit{flat}, if $T_N: M \mapsto M \otimes_A N$ is \textit{exact}.
\end{definition}

\begin{theorem}The following are equivalent:
    \begin{itemize}
        \item $N$ is flat
        \item $T_N$ is exact
        \item If $f:M \rightarrow M'$ is injective, then $T_N(f)$ is injective
        \item If $f:M \rightarrow M'$ is injective and $M,M'$ finitely generated, then $T_N(f)$ is injective
    \end{itemize}
\end{theorem}
\textit{Proof:}
$(i) \iff (ii)$ by definition, $(ii)\iff (iii)$ by right-exactness, $(iii)\implies (iv)$ clear, $(iv)\implies (iii):$ Let $f: M' \rightarrow M$ be injective and let $u = \sum x_i \otimes y_i \in \ker(f\otimes 1)$, so that $0=\sum f(x'_i)\otimes y_i \in M\otimes N$. Let $M'_0$ be the submodule generated by the $x'_i$ and let $u_0$ denote $\sum x'_i \otimes y_i$ as an element of $M'_0 \otimes N$. By \textit{some lemma} there exists a finitely generated submodule $M_0$ of $M$ containing $f(M'_0)$ and such that $\sum f(x'_i)\otimes y_i = 0$ as an element of $M_0 \otimes N$.

\subsection{Finitely generated modules}

\begin{theorem}[Nakayama's Lemma]
    Let $M$ be a finitely generated $A$-module and $\fa$ an ideal of $A$ contained in the Jacobson radical $\mathfrak{R}$ of $A$. Then $\fa M = M$ implies $M=0$.
\end{theorem}

\marginnote[-30pt]{Jacobson radical --- intersection of all the maximal ideals of $A$.}

\begin{lemma}
    Let $M$ be a finitely generated $A$-module and let $\fa$ be an ideal of $A$ such that $\fa M = M$. Then there exists $x\equiv 1\bmod{\fa}$ such that $xM=0$.
\end{lemma}

\begin{lemma}
    Let $M$ be a finitely generated $A$-module, let $\fa$ be an ideal of $A$, and let $\phi$ be an $A$-module endomorphism of $M$ such that $\phi(M)\subseteq \fa M$. Then $\phi$ satisfies and equation of the form 
    $$\phi^n + a_1 \phi^{n-1}+\ldots + a_n = 0$$
    where the $a_i$ are in $\fa$.
\end{lemma}

\textit{Proof:} Let $M=(x_1\ldots x_n)$. Then $\phi(x_i) \in \fa M$, so that we have say $\phi(x_i)=\sum^n_{j=1}a_{ij}x_j$ for $1\leq i \leq n, a_{ij}\in \fa$ because $\phi(x_i)$ is still a linear combination of generators of $M$. By Cayley-Hamilton, $\phi$ satisfies its own characteristic equation, hence the statement.

\section{Commutative Algebra I}
\subsection{Integral Dependence and Going-Up Theorem}

The general setting is that we have commutative unital ring $A$ and its extension ring $B$. We want to understand the map $$\Spec(B)\rightarrow\Spec(A), \quad \fq \mapsto \fq \cap A$$
In field theory, the main objects were finite and algebraic field extensions. Integral ring extensions are their ring theory counterpart. 

\begin{theorem}[Going-Up Theorem]
    Let $A\subseteq B$ be an integral extension. Suppose $$\fq_1 \subseteq \ldots \subseteq \fq_n$$ is a prime ideal chain in $B$. Suppose $$\fp_1 \subseteq \ldots \subseteq \fp_m$$ is a (longer) prime ideal chain in $A$ such that $\forall i \leq n: \fp_i = \fq_i \cap A$.
    Then there exists a continuation $\fq_{n+1} \subseteq \ldots \subseteq \fq_{m}$ of prime ideals in $B$.
\end{theorem}

\marginnote[-60pt]{As I understand it, this means $\{\text{prime ideal chains in } A\} \longleftrightarrow \{\text{prime ideal chains in } B\}$.
Equivalently, $A$ and $B$ have the same Krull dimension.}

\subsection{The Spectrum, Again}

\section{Intro to Algebraic Geometry}

\subsection{Noether Normalization and Hilbert's Nullstellensatz}

\begin{theorem}[Noether normalization theorem]
    Let $k$ be a field. Let $A$ be a finitely generated $k$-algebra. Then there exist algebraically independent $\{x_1\ldots x_n\}\in A$ such that $A$ is finite over $k[x_1\ldots x_n]$.
\end{theorem}

\begin{theorem}[Hilbert's Nullstellensatz]
    Let $k$ be a field. Let $A$ be a finitely generated $k$-algebra, and let $\fm \subseteq A$ be a maximal ideal. Then $A/\fm$ is a finite field extension of $k$.
\end{theorem}

\begin{theorem}[Weak Nullstellensatz]
    Let 
    \begin{itemize}
        \item $k$ be an algebraically closed field,  
        \item $f_1\ldots f_m\in k[X_1\ldots X_n]$ arbitrary,
        \item $A:= k[X_1\ldots X_n] / (f_1\ldots f_m).$
    \end{itemize}
    Then there exists a solution $x\in k^n \iff (f_1\ldots f_m) \neq k[X_1\ldots X_n].$ Moreover, there exist infinitely many solutions iff $\dim_k(A)=\infty$.
\end{theorem}

\begin{theorem}[Bezout's theorem]
    Let
    \begin{itemize}
        \item $k$ be an algebraically closed field,
        \item $f,g\in k[X,Y]$ be of degrees $n,m$,
        \item $S:=\{(x,y)\in k^2 \mid f(x,y)=g(x,y)=0\}$ be their solution set,
        \item $A:=k[X,Y]/(f,g)$. 
    \end{itemize}
    
    Then the following holds: 
    
    $S$ is infinite $\iff f,g$ have a common non-trivial factor

    $S$ finite $\implies$ $|S|\leq \dim_k(A) \leq nm$. 
\end{theorem}

\subsection{Algebraic Sets and Ideals}

\begin{definition}
    \begin{itemize}
        \item Algebraic set $Z\subseteq k^n$ 
        
        $\iff$ exists some subset $S$ of $k[X_1\ldots X_n]$, such that $\forall z\in Z: \forall f \in S f(z)=0$ 
        
        $\iff$ definable by some polynomial formula. 
        
        Der Unterschied zu semi-algebraischen Mengen ist, dass semi-algebraische Mengen durch Ungleichungen definierbar sind.  
        
        \item Vanishing set $Z(S)$

        $\iff$ Menge der Nullstellen von $S$. 
        
        \item Zariski topology

        $\iff$ Algebraic sets form closed sets on $k^n$.
        
        \item Vanishing ideal

        $\iff$ Given (any) set $Y\subset k^n$, we define the vanishing ideal $I(Y)$ as the set of functions equal to zero for all $y\in Y$.
        
        \item Radical 

        $\iff$ Any power $x^n \in \fa \implies x \in \fa$ for all $x\in A$.
    \end{itemize}
\end{definition}

\begin{theorem}[Hilbert's Nullstellensatz, Algebraic Geometry]
    Let $k$ be an algebraically closed field. Then $Z$ and $I$ define mutually inverse bijections between algebraic subsets of $k^n$ and radical ideals in $k[X_1\ldots X_n]$ via $Z\mapsto I(Z)$ and $Z(\fa)\mapsfrom \fa$. 
    
    More generally, we have $Z(I(Z))=Z$ for all algebraic subsets $Z\subseteq k^n$ and $I(Z(\fa))=\sqrt{\fa}$ for all ideals $\fa \subseteq k[X_1\ldots X_n]$.
\end{theorem}

\begin{definition}
    Jacobson ring
\end{definition}

\subsection{Krull dimension}

\begin{theorem}[Krull's principal ideal theorem]
    f
\end{theorem}

\begin{theorem}
    Let $k$ be a field. Then $\dim(k[X_1\ldots X_n])=n$.
\end{theorem}

\subsection{Transcendence Degree}

\begin{definition}
    Transcendence basis, transcendence degree
\end{definition}

\subsection{Irreducible components, Minimal Prime Ideals}

\begin{definition}
    Irreducible algebraic set, irreducible component
\end{definition}

\subsection{Krull's Principal Ideal Theorem}

\section{Intro to Algebraic Number Theory}

\subsection{Integral closure}

\begin{definition}
    Algebraic number field, ring of algebraic integers

    Norm, trace, characteristic polynomial
\end{definition}

\subsection{Localization and Discrete Valuation Rings}

\subsection{Dedekind Rings}

\subsection{Fractional Ideals}

\subsection{Ideal Class Group}

\subsection{The Splitting of Primes}

\subsection{Quadratic Norm Equations}

\subsection{Hilbert Class Fields and a Theorem of Gauss}
Not important.





\end{document}


TO-DO:
Krull's PID Theorem
Hilbert's Nullstellensatz
Nakayama's lemma